\documentclass{llncs}
\usepackage{graphicx}
\usepackage{listings}

\begin{document}
\title{Finding regional extrema - methods and performance}
\author{Richard Beare{$^1$} \and Gaetan Lehmann{$^2$}}
\institute{Department of Medicine, Moansh University, Australia
\and Biologie du Developpement et de la Reproduction, INRA de Jouy-en-Josas (France)}
\maketitle

\begin{abstract}
Finding regional extrema of images is an important step in a number of
important morphological algorithms, such as some versions of the
watershed transform. Regional extrema may also be important cues of
other tasks, such as splitting objects based on distance transform
information\footnote{This work started because we were trying to
compare different watershed algorithms}. This report provides an
overview of the methods available in ITK and compares the performance
with a new filter.
\end{abstract}

\section{Introduction}
There are two classes of regional extrema - regional maxima and
regional minima. Regional maxima are flat zones that are connected to
pixels of lower value while regional minima are flat zones that are
connected to pixels of higher value. The notion of higher and lower is
obviously critcal to the definition of regional maxima and minima,
which means that it doesn't make sense to apply the definitions
directly to vector images.

\section{Filters currently available in ITK}
Regional extrema are already used inside the ITK watershed
implementation and may be computed using {\em itkHConvexImageFilter}
and {\em itkHConcaveImageFilter}. 

The regional extrema computed in the watershed are not available
externally (which could obviously be changed). The technique used in
the watershed filters is based on a labelling approach, very similar
to that used in the {\em ConnectedComponentImageFilter}, where each
region of uniform intensity is labelled and checked to determine
whether it is an extrema or plateau.

The convex and concave filters are based on the principles of
morhpological reconstruction. Regional minima are located as follows:
create a marker image by adding $h$ to the original, carry out a
reconstruction by erosion of the marker image using the original as
the control surface, and then, finally, take the difference between
them. If $h=1$ then any non-zero pixels are regional minima, if $h$ is
greater than one then the non zero pixels are minima that are at least
$h$ below the lowest surrounding ridge point. The value $h$ is often
referred to as the {\em dynamic} of the minima, and is often used to
give an indication of the importance of the regional minima.

The concave and convex filters are therefore capable of providing more
information than just the location and value of the regional minima -
they also provide some clues about the local topology. However this
additional information comes at a computational cost as well as a
restriction on the type of images to which the technique can be
applied. The need to add or subtract $h$, rather than relying on
equality, means that it is only possible to find extrema when the
smallest step in image intensity is known. This is difficult to do
sensibly on floating point image types.

\section{{\em itkValuedRegionalMaximaImageFilter} and {\em itkValuedRegionalMinimaImageFilter}}
The new filters use a simple flooding approach to find regional
extrema. They produce an output image with the non minima (maxima) set
to the maximum (minimum) of the pixel type. The minima or maxima
retain their original value. 
The flooding approach is very simple, and proceeds, for minima detection, as follows:
\begin{itemize}
\item Copy the input to the output.
\item Visit each pixel of the input image. 
   \begin{itemize}	
   \item If the the corresponding output 
	 value is not maximal (meaning this pixel hasn't already been
      visited) then check all the neighbors. 
    \item If any of the neighboring grey levels are less than the current pixel 
      value, then this pixel cannot be a regional minima.
    \item Flood fill the region, in the output image, with the same grey level 
      as the current pixel that contains the current pixel, with the
      maximal value for the pixel type.
    \end{itemize}
\item Go to next input pixel.
\end{itemize}

This algorithm requires that the image border is preset to the maximum
value for the pixel type, which is done using a constant boundary
condition.

If an image is a constant it will be returned as either a regional
minimum or a regional maximum, depending on the filter used, because
the situation is not properly defined. The filter will set a flag if
this situation arises.

\section{Performance}
A timing test comparing performance on a $371 \times 371 \times 34$
image showed that the flooding approach was significantly faster than
the reconstruction approach. The results achieved on an AMD Athlon 64
3000+ (1800MHz) with 512Kb cache and 4GB ram were:
\begin{verbatim}
> ./perf3D ESCells.img
#F      concave rmin
0       20.674  3.54
1       24.456  5.074
\end{verbatim}
This is an approximately 5 fold reduction in computation time. Note
that the left-most column indicates whether the filters were run in
fully connected(1) or face connected mode.

There are a few approaches that might be taken to improve performance
further. For example, that standard face calculator might be used to
reduce the boundary condition check. Alternatively, if a copy of the
input image is made, and the border of that is set to the maximum
value, then there is no need for a boundary check at all. The cost of
this is a missing border region and probably an extra copy of the
image.

\section{Results - just to prove that it runs}
\begin{figure}[htbp]
\begin{center}
\includegraphics[scale=3]{reg_min_ov.png}
\caption{Regional minima found using the flooding method}
\end{center}
\end{figure}


\section{Comparison of operations}
So which method is ``best''? Well, they are all a bit different, so
the choice will be a bit application dependent. The labelling approach
isn't currently available as standalone filter, so it is difficult to
judge performance. The algorithm requires several steps, which may
slow it down, but some of the steps are certain to visit the pixels in
raster order, which can be a big advantage. If the regional extrema
are going to be labelled anyway then the performance may be
competitive. The reconstruction based approach is considerably slower,
but is able to do a lot more. If you are interested in image dynamics
rather than regional extrema then this is the only way to go. If you
definitely want extrema then the new filter is currently the fastest
way to go.

\section{Sample code}
The following code is from {\tt vrmin.cxx}, and compares the operation
of ValuedRegionalMinimaImageFilter and HConcaveImageFilter.

\lstset{language=C,  basicstyle=\small}
\begin{lstlisting}
// a test routine for regional extrema using flooding
#include "itkValuedRegionalMinimaImageFilter.h"
#include "itkHConcaveImageFilter.h"
#include "itkMaximumImageFilter.h"
#include "itkInvertIntensityImageFilter.h"
#include "itkImageFileReader.h"
#include "itkImageFileWriter.h"
#include "itkCommand.h"
#include <itkRescaleIntensityImageFilter.h>
#include <itkAndImageFilter.h>
#include "itkCommand.h"

template < class TFilter >
class ProgressCallback : public itk::Command
{
public:
  typedef ProgressCallback   Self;
  typedef itk::Command  Superclass;
  typedef itk::SmartPointer<Self>  Pointer;
  typedef itk::SmartPointer<const Self>  ConstPointer;

  itkTypeMacro( IterationCallback, Superclass );
  itkNewMacro( Self );

  /** Type defining the optimizer. */
  typedef    TFilter     FilterType;

  /** Method to specify the optimizer. */
  void SetFilter( FilterType * filter )
    {
    m_Filter = filter;
    m_Filter->AddObserver( itk::ProgressEvent(), this );
    }

  /** Execute method will print data at each iteration */
  void Execute(itk::Object *caller, const itk::EventObject & event)
    {
    Execute( (const itk::Object *)caller, event);
    }

  void Execute(const itk::Object *, const itk::EventObject & event)
    {
    std::cout << m_Filter->GetNameOfClass() << ": " 
	<< m_Filter->GetProgress() << std::endl;
    }

protected:
  ProgressCallback() {};
  itk::WeakPointer<FilterType>   m_Filter;
};


int main(int, char * argv[])
{
  const int dim = 2;
  
  typedef unsigned char PType;
  typedef itk::Image< PType, dim > IType;

  typedef itk::ImageFileReader< IType > ReaderType;
  ReaderType::Pointer reader = ReaderType::New();
  reader->SetFileName( argv[2] );

  typedef itk::ValuedRegionalMinimaImageFilter< IType, IType > FilterType;
  FilterType::Pointer filter = FilterType::New();
  filter->SetInput( reader->GetOutput() );
  filter->SetFullyConnected( atoi(argv[1]) );

  typedef ProgressCallback< FilterType > ProgressType;
  ProgressType::Pointer progress = ProgressType::New();
  progress->SetFilter( filter );

  typedef itk::ImageFileWriter< IType > WriterType;
  WriterType::Pointer writer = WriterType::New();
  writer->SetInput( filter->GetOutput() );
  writer->SetFileName( argv[3] );
  writer->Update();


  // produce the same output with other filters
  typedef itk::HConcaveImageFilter< IType, IType > ConcaveType;
  ConcaveType::Pointer concave = ConcaveType::New();
  concave->SetInput( reader->GetOutput() );
  concave->SetFullyConnected( atoi(argv[1]) );
  concave->SetHeight( 1 );

  // concave gives minima with value=1 and others with value=0
  // rescale the image so we have minima=255 other=0
  typedef itk::RescaleIntensityImageFilter< IType, IType > RescaleType;
  RescaleType::Pointer rescale = RescaleType::New();
  rescale->SetInput( concave->GetOutput() );
  rescale->SetOutputMaximum( 255 );
  rescale->SetOutputMinimum( 0 );

  // in the input image, select the values of the pixel at the minima
  typedef itk::AndImageFilter< IType, IType, IType > AndType;
  AndType::Pointer a = AndType::New();
  a->SetInput(0, rescale->GetOutput() );
  a->SetInput(1, reader->GetOutput() );

  // all pixel which are not minima must have value=255
  // get the non minima pixel by inverting the rescaled image
  // minima have value=0 and non minima value=255
  typedef itk::InvertIntensityImageFilter< IType, IType > InvertType;
  InvertType::Pointer invert = InvertType::New();
  invert->SetInput( rescale->GetOutput() );

  // get the highest value from "and" and from invert. 
  // The minima have value>=0 in and image
  // and the non minima have a value =0. In invert, 
  // the non minima have a value=255 and the minima
  // a value=0
  typedef itk::MaximumImageFilter< IType, IType, IType > MaxType;
  MaxType::Pointer max = MaxType::New();
  max->SetInput(0, invert->GetOutput() );
  max->SetInput(1, a->GetOutput() );

  WriterType::Pointer writer2 = WriterType::New();
  writer2->SetInput( max->GetOutput() );
  writer2->SetFileName( argv[4] );
  writer2->Update();

  // to verify if the image is flat or not
  filter->Print( std::cout );

  return 0;
}

\end{lstlisting}


\end{document}

