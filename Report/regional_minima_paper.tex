\documentclass{llncs}
\usepackage{graphicx}

\begin{document}
\title{Finding regional extrema - methods and performance}
\author{Richard Beare{$^1$} \and Gaetan Lehmann{$^2$}}
\institute{Department of Medicine, Moansh University, Australia
\and Biologie du Developpement et de la Reproduction, INRA de Jouy-en-Josas (France)}
\maketitle

\begin{abstract}
Finding regional extrema of images is an important step in a number of
important morphological algorithms, such as some versions of the
watershed transform. Regional extrema may also be important cues of
other tasks, such as splitting objects based on distance transform
information. This report provides an overview of the methods available
in ITK and compares the performance with a new filter.
\end{abstract}

\section{Introduction}
There are two classes of regional extrema - regional maxima and
regional minima. Regional maxima are flat zones that are connected to
pixels of lower value while regional minima are flat zones that are
connected to pixels of higher value.

perhaps mention ordering and how it doesn't really apply to vector pixels.

\section{Filters in ITK}
Watershed -- labelling algorithm.

Reconstruction - relationship with image dynamics.

\section{Comparison of operations}
Labelling scheme will produce a labelled image, which might be
preferrable if the next step is some sort of shape statistics
analysis, or a marker based procedure.

Reconstruction method uses a ``substract 1'' step, which isn't well
defined for all pixel types. However it does allow the option of
placing a significance level on a regional extrema via dynamics.

Flooding method is pixel type independent. It produces an output where
all non extrema pixels are set to the opposite extreme.

\end{document}

